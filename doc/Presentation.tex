\documentclass{beamer}
\usepackage[utf8]{inputenc}
\usepackage[french,frenchb,francais]{babel}
\usetheme{Goettingen}
\usecolortheme{seahorse}
\setbeamercolor*{titlelike}{use=structure,fg=structure.fg}
\useoutertheme[height=0pt]{sidebar}
\setbeamercolor{sidebar}{parent=palette primary}
{\usebeamercolor{palette quaternary}}
{\usebeamercolor{palette primary}}
\author[\textsc{Phélipot\and Kotnik\and Noro\and Tiona\and Rimoux}]{\textsc{Phélipot Pascal\\ Kotnik Guillaume\\ Noro Geoffrey\\ Ralijaona Tiona\\ Rimoux Quentin}}
\title[Jeu Snake]{Projet Java : Jeu Snake}
\date{Semestre 2}
\subject{Snake}

\setbeamertemplate{navigation symbols}{}

\begin{document}
\begin{frame}
\maketitle
\end{frame}

\begin{frame}
\frametitle{Sommaire}

\tableofcontents	
\end{frame}

\section{Introduction}
\begin{frame}
\frametitle{\insertsection}

Dans le cadre de notre projet tutoré il nous a été demandé de développer un jeu du serpent.\\
Nous avons donc décidé de se répartir les différentes taches de notre projet :\\
\indent - Tiona et Guillaume => Développement du menu et son interaction avec le jeu \\
\indent - Pascal et Geoffrey => Développement du jeu et de ses fonctionalités \\
\indent - Quentin => Developpement de l'IA
\end{frame}

\section{Interface}
\subsection{Menus}
\begin{frame}
\frametitle{\insertsubsection}
\framesubtitle{\insertsection}

\end{frame}
\subsection{Options}
\begin{frame}
\frametitle{\insertsubsection}
\framesubtitle{\insertsection}

\end{frame}
\subsection{Scores}
\begin{frame}
\frametitle{\insertsubsection}
\framesubtitle{\insertsection}

\end{frame}

\section{Fonctionnement}
\subsection{Grille}
\subsubsection{Cases}
\begin{frame}
\frametitle{\insertsubsubsection}
\framesubtitle{\insertsubsection}
\framesubsubtitle{\insertsection}
Pour la réalisation de ce jeu nous avions besoins absolument d'une grille.\\
Les cases de cette grille peuvent être :\\
\indent - Une case libre\\
\indent - Un mur\\
\indent - Un fruit\\
\indent - Une partie du serpent

\end{frame}
\subsubsection{Fruits}
\begin{frame}
\frametitle{\insertsubsubsection}
\framesubtitle{\insertsubsection}
\framesubsubtitle{\insertsection}
Il devait être possible de generer un certain nombre de fruits et de le maintenir.\\
De plus un fruit devait aussi être possible que le serpents la différencie des cases libres en passant dessus.\\
De plus nous avons implémenté des fruits différents avec un gain de score différents et de taille :\\
\indent - la pomme  (25)\\
\indent - la poire  (35)\\
\indent - la cerise (50)\\
\indent - la banane (75)

\end{frame}
\subsection{Serpents}
\subsubsection{Contrôles des joueurs}
\begin{frame}
\frametitle{\insertsubsubsection}
\framesubtitle{\insertsubsection}
\framesubsubtitle{\insertsection}
Nous générons deux serpents de manières automatique au début du jeu.\\
Ces serpents après chaque fruits manger devait augmenter sa vitesse.\\
Pour controler les serpents on utilise des listeners ( écouteurs ) des touches du clavier.\\
Pour le joueur 1 :\\
\indent - fleche haut\\
\indent - fleche bas\\
\indent - fleche droite\\
\indent - fleche gauche\\
Pour le joueur 2 :\\
\indent - E\\
\indent - D\\
\indent - S\\
\indent - F\\

\end{frame}
\subsubsection{Contrôle par l'ordinateur}
\begin{frame}
\frametitle{\insertsubsubsection}
\framesubtitle{\insertsubsection}
\framesubsubtitle{\insertsection}
Nous voulions aussi integrer une IA contre laquelle l'utilisateur pourrait jouer.\\
Pour cela nous devions toujours cherchez le fruit le plus proche du serpents pourqu'il s'y dirige.\\

\end{frame}
\subsection{Design}
\begin{frame}
\frametitle{\insertsubsection}
\framesubtitle{\insertsection}
Pour les parties du serpents :\\
(images de la tête, d'une partie du corps, de la queue)\\
Pour les fruits :\\
(images des 4 fruits différents)

\end{frame}


\section{Conclusion}
\subsection{Possibilités d'améliorations}
\begin{frame}
\frametitle{\insertsubsection}
\framesubtitle{\insertsection}
Les améliorations possibles :\\
\indent - L'IA ( elle peut toujours être améliorer )\\
\indent - Un mode multijoueur en réseau\\
\indent - Un changement de la taille de la grille et la possibilité de la rendre rectangulaire

\end{frame}
\subsection{Que nous a apportez ce projet}
\begin{frame}
\frametitle{\insertsubsection}
\framesubtitle{\insertsection}
A l'aide de se projet nous avons appris à travailler en équipe, à nous entraider essayant partageant au maximum nos connaissances et en ne laissant jamais un membre du groupe avec ses problêmes mais en essayant de l'aider au mieux.\\
Nous avons aussi appris à utiliser GitHub afin de mieux coordonner nos travaux et de tous pouvoir intervenir librement sur le projet.\\
Ce projet nous a aussi permi de voir que notre organisation peut être encore améliorer afin d'être encore plus efficace.

\end{frame}

\begin{frame}
\begin{center}
	Merci de votre attention
\end{center}
\end{frame}

\end{document}